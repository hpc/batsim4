%%This is a very basic article template.
%%There is just one section and two subsections.
\documentclass[titlepage]{article}
\author{Craig Walker}
\title{CCU-LANL-Batsim-Framework-Guide}
\usepackage{amsmath,amssymb,amsfonts}
\usepackage{helvet}
\usepackage{fontspec}
\usepackage{fontawesome}
\usepackage{listings}
\usepackage{soul}
\usepackage{xparse}
\usepackage{fvextra}


\usepackage{xunicode}
\usepackage{xltxtra}
\usepackage{xcolor}
\usepackage{array}
\usepackage{hhline}
\usepackage[bookmarks,bookmarksopen,bookmarksdepth=5,colorlinks=true]{hyperref}
\usepackage{bookmark}
\hypersetup{colorlinks=true, linkcolor=blue, citecolor=blue, filecolor=blue, urlcolor=blue}
\usepackage{polyglossia}
\usepackage[most]{tcolorbox}
\usepackage[margin=0.5in]{geometry}
\usepackage{lstlinebgrd}
\usepackage{sectsty}
\usepackage{titlesec}
\usepackage{fancyvrb,newverbs,xcolor}
\setdefaultlanguage[variant=american]{english}
% Text styles
\definecolor{hi-light-yellow}{HTML}{FFFF6D}
\definecolor{hi-light-orange}{HTML}{FFE994}
\definecolor{hi-light-pink}{HTML}{FFD8CE}
\definecolor{hi-light-purple}{HTML}{E0C2CD}
\definecolor{hi-light-green}{HTML}{E8F2A1}
\definecolor{hi-yellow}{HTML}{FFFF00}
\definecolor{lightest-gray}{gray}{.97}
\definecolor{codelist}{RGB}{237, 204, 152}
\colorlet{codelist-light}{codelist!50}
\colorlet{terminal-light}{black!15}
\definecolor{code-blue}{RGB}{89,131,176}
\definecolor{easter-blue}{RGB}{0,200,255}
\definecolor{myLightBlue}{RGB}{0,100,255}
\colorlet{lightest-blue}{easter-blue!10}
\colorlet{explanation-code}{myLightBlue!23}
% New Commands
\let\oldsection\section
\newcommand\spsection{\oldsection} % same page section
\renewcommand\section{\clearpage\oldsection}
\makeatletter
\titleformat*{\section}{\LARGE\bfseries}
\titleformat*{\subsection}{\Large\bfseries}
\titleformat*{\subsubsection}{\large\bfseries}
\titleformat*{\paragraph}{\normalsize\bfseries}
\titleformat*{\subparagraph}{\small\bfseries}
\renewcommand\paragraph{\@startsection{paragraph}{4}{\z@}{-3.25ex \@plus1ex \@minus.2ex}{10pt}{\sffamily\normalsize\bfseries}}
\renewcommand\subparagraph{\@startsection{subparagraph}{4}{\z@}{-3.25ex \@plus1ex \@minus.2ex}{10pt}{\sffamily\small\bfseries}}
%\definecolor{explanation-code}{RGB}{50,150,255}

\newenvironment{expverbatim}
 {\SaveVerbatim{cverb}}
 {\endSaveVerbatim
  \flushleft\fboxrule=0pt\fboxsep=.5em
  \colorbox{explanation-code}{%
    \makebox[\dimexpr\linewidth-2\fboxsep][l]{\BUseVerbatim{cverb}}%
  }
  \endflushleft
}
% types of explanation boxes
\newcommand\infoC{\faInfoCircle}
\newcommand\alertC{\faExclamationCircle}
\newcommand\alertT{\faExclamationTriangle}
%
\AfterEndEnvironment{explanation}{\color{black}}
\AfterEndEnvironment{itemize}{\color{black}}
\AfterEndEnvironment{enumerate}{\color{black}}
\AfterEndEnvironment{description}{\color{black}}

%\renewcommand{\familydefault}{\sfdefault}
\newenvironment{regular}{\color{black}}{}
\usepackage{marvosym}


%-------------------------------------------------------------------
%                            Background Color Ranges
%-------------------------------------------------------------------
\usepackage{pgf}
\usepackage{pgffor}

\makeatother
\newcommand\myIfRange[3]{%
\ifnum\value{lstnumber}>\numexpr#1-1\relax
	\ifnum\value{lstnumber}<\numexpr#2+1\relax
		\color{#3}
	\fi
\fi}
\newcommand\myIf[2]{%
\ifnum\value{lstnumber}=#1
	\color{#2}
\fi}
\newcommand\myStart{\color{lightest-gray}}
\newcommand\myTitle{What This Code Does}
					
					
%--------------------------------------------------------------------

%--------------------------------------------------------------------
						
\NewTCBListing{code}{ O{} }{%
  breakable,
  left=5pt,
  colback=codelist!50,
  colframe=codelist!50,
  arc=2pt,
  boxrule=0pt,  
  listing only,
  coltext=code-blue,
  listing options={showstringspaces=false,language=c,style=CODE-STYLE,tabsize=1,#1}
}
\NewTCBListing{terminal}{ O{style=TERMINAL} }{%
  top=-1mm,
  bottom=-1mm,
  before skip=10pt,
  after skip=10pt,
  colback=black!15,
  colframe=black!15,
  arc=2pt,
  boxrule=0pt,
  listing only,
  listing options={tabsize=1,#1}
}
\NewTCBListing{explanation}{ O{style=EXPLANATION-STYLE} m m}{%
	enhanced,
	colback=myLightBlue!30,
	colframe=myLightBlue!30!black!90,
	coltitle=lightest-blue!50,
	fonttitle={\fontsize{20}{30}\bfseries},
	listing only,
	breakable,
	arc=5pt,
	title={#2 \large \quad #3},
	listing options={showstringspaces=false,showtabs=false,escapechar=@|,#1}
}
\setlength\parindent{0pt}
\setcounter{tocdepth}{5} 
\setcounter{secnumdepth}{5}
\definecolor{myTerminalRed}{RGB}{201, 60, 73}
\definecolor{myExplanationRed}{RGB}{160,40,100}
\setmonofont{DejaVu Sans Mono}


\makeatletter

\newcommand\mytokens[3]{\mytokenshelp{#1}{#2}#3 \relax\relax}
\def\mytokenshelp#1#2#3 #4\relax{\allowbreak\grayspace\tokenscolor{#1}{#2}{#3}\ifx\relax#4\else
 \mytokenshelp{#1}{#2}#4\relax\fi}
\newcommand\tokenscolor[3]{\colorbox{#1}{\textcolor{#2}{%
  \small\ttfamily\mystrut\smash{\detokenize{#3}}}}}
\def\mystrut{\rule[\dimexpr-\dp\strutbox+\fboxsep]{0pt}{%
 \dimexpr\normalbaselineskip-2\fboxsep}}
\def\grayspace{\hspace{0pt minus \fboxsep}}
\newcommand{\lstTerm}[1]{\mytokens{black!15}{myTerminalRed}{#1}}
\newcommand{\lstFolder}[1]{\mytokens{black!15}{blue!70}{#1}}
\newcommand{\lstCode}[1]{\mytokens{codelist!50}{code-blue}{#1}}

%\NewDocumentCommand{\lstCode}{v}{\sethlcolor{codelist-light}\texthl{\Verb{#1}}}
%\newcommand{\lstCode}[1]{\hl{\Verb#1}}
%\soulregister\lstCode1
%\newcommand{\lstCode}[1]{%
%  \edef\hverb@tmp{#1}%
%  \expandafter\hl\expandafter{\hverb@tmp}}
\makeatother






\begin{document}
\sffamily
\allsectionsfont{\normalfont\sffamily\bfseries}


\lstdefinestyle{EXPLANATION-STYLE}
{
 	columns=flexible,
 	breaklines=true,
    basicstyle={\normalsize\color{black}\sffamily}
}
\lstdefinestyle{TERMINAL}
{
 	columns=fullflexible,
 	breaklines=true,
	backgroundcolor=\color{black!15},
    basicstyle=\small\color{myTerminalRed}\ttfamily
}
\lstdefinestyle{TERMINAL-FLEX}
{
 	columns=flexible,
 	breaklines=true,
	backgroundcolor=\color{black!15},
    basicstyle=\small\color{myTerminalRed}\ttfamily
}
\lstdefinestyle{CODE-STYLE}
{
	columns=flexible,
	breaklines=true,
	escapechar=@|,
	backgroundcolor=\color{codelist!50},
    basicstyle=\small\ttfamily
}


\maketitle
\hypersetup{linkcolor=blue,urlcolor=blue,anchorcolor=blue}
\pdfbookmark[section]{Table Of Contents}{1}
\tableofcontents
\pagebreak

\section{Config}
\begin{regular}

To use the batsim framework you must first make a config file.
You can see how to make this file using basefiles/generate\_config.py

\begin{terminal}
python3 generate_config.py --help
\end{terminal}

\begin{terminal}
Usage:
    generate_config.py --config-info <type>
    generate_config.py -i FILE -o PATH [--basefiles PATH] [--output-config] [--increase-heldback-nodes]

Required Options 1:
    --config-info <type>                Display how the json config is supposed to look like
                                        as well as how each part of it can look like.
                                        <type> can be:
                                                general | sweeps |
                                                node-sweep | SMTBF-sweep | checkpoint-sweep | checkpointError-sweep | performance-sweep |
                                                grizzly-workload | synthetic-workload |
                                                input-options | output
Required Options 2:
    -i <FILE> --input <FILE>            Where our config lives
    -o <PATH> --output <PATH>           Where to start outputing stuff
                                        
Options:
    --basefiles <PATH>                  Where base files go.  Make sure you have a 'workloads' and 'platforms' folder
                                        in this path.
                                        [default: ~/basefiles/]

    --output-config                     If this flag is included, will output the input file to --output directory
                                        as --input filename

    --increase-heldback-nodes           If this flag is included, will treat heldback nodes as additional nodes

\end{terminal}

As you can see you can get --config-info  so type:
\begin{terminal}
python3 generate_config.py --config-info general
\end{terminal}
\begin{terminal}
 The general format of a config file:
    
    {       <------------------------------------   Opening curly brace to be proper json
    
        "Name1":{       <------------------------   The name of an experiment comes first.  You can have multiple experiments
                                                    in one config file and each will end up in it's own folder under the --output folder.
                                                    Notice the opening and closing curly brace.  Make sure you put a comma after the closing
                                                    curly brace if you plan on having another experiment in the same config file
                
                "input":{    <-------------------   Always make sure you have an input and an output in your experiment
                
                    "node-sweep":{  <------------   It is MOST advisable to always start with a node-sweep.  All other sweeps can come after this one
                    
                    },
                    "synthetic-workload":{ <-----   Always include either a synthetic-workload or a grizzly-workload after your sweeps
                    
                    },
                    "option":value,        <-----   Include any options that will affect all of the jobs on the outside of any sweep or workload
                
                },    <--------------------------   Make sure you separate your input options with commas, but also remember to separate input
                                                    and output with a comma
                "output":{   <-------------------   Again, always make sure you have an input and output in your experiment
                
                    "option":value,   <----------   Output is a bit simpler than input.  Just make sure it is valid json
                    "option":value
                
                }
        
        
        },     <---------------------------------   This closes the experiment and here we have a comma because we included another experiment "Name2"
        "Name2":{
            "input":{
            
                ...  <--------------------------    Make sure you replace this ellipsis with at least:
                                                        * a node-sweep
                                                        * a workload
            },
            "output":{
            
                ...  <--------------------------    You should replace ellipsis with at least:
                                                        * "AAE":true | "makespan":true
                
            }    <------------------------------    Close output
        }  <------------------------------------    Close "Name2"          
    }  <----------------------------------------    Close json

\end{terminal}




\subsection{Run Config}
You will then want to run the config, for example:
\begin{terminal}
python3 generate_config.py -i /ac-project/cwalker/basefiles/$file1 -o /ac-project/cwalker/experiments/$folder1 --basefiles ${basefiles}  --output-config

# Here $file1 is the json config file we already mentioned
# $folder1 is where you want the output to go for all the simulations corresponding to this config file
# $basefiles is where all the scripts are located, including generate_config.py
\end{terminal}
\section{Run Experiments}
Next you will want to run the experiments
\begin{terminal}
python3 run-experiments.py -i /ac-project/cwalker/experiments/$folder1  --time 1200 --sim-time-minutes 1200 --socket-start 100003

--time is in minutes as well as --sim-time-minutes
--time is for slurm wall time
--sim-time-minutes is a setting for batsim for how long the sim should go.  We used to set it to a super high value, I cut back and set it for the walltime in case there was a problem with the simulation so SLURM would be happier.

\end{terminal}

run-experiments.py will:
\begin{itemize}
\item sbatch real\_start.py for each simulation, which will:
\begin{itemize}
  \item generate a yaml for the simulation with robin
  \item run the simulation with robin
  \item when finished the simulation, run post-processing.py
\end{itemize}
\end{itemize}

\section{Aggregate Results}
You will probably want to aggregate all the results into one csv file.
Use aggregate\_makespan.py for this.  It is our main aggregater, even though it was originally used just for makespan.

You can get its usage using this:
\begin{terminal}
python3 aggregate_makespan.py --help
\end{terminal}
\begin{terminal}
Usage:
    aggregate-makespan.py -i FOLDER [--output FOLDER] [--start-run INT] [--end-run INT]

Required Options:
    -i FOLDER --input FOLDER    where the experiments are

Options:
    -o FOLDER --output FOLDER   where the output should go
                                [default: input]
    --start-run INT             only include runs starting at start-run
                             
    --end-run INT               only include runs ending at and including end-run

\end{terminal}
I usually run it like this:
\begin{terminal}
python3 aggregate_makespan.py -i ~/ac/experiments/$folder1

#where folder1 equals the location set when you ran generate_config.py
\end{terminal}
This will output a file total\_makespan.csv as well as raw\_total\_makespan.csv \\

The difference is that raw gathers every makespan.csv file into one file

wheras total\_makespan.csv will average all runs together and output the averages for each experiment.
If you want to graph variance and such you will need raw \\

In order to graph single job variances you will need to use the out\_jobs.csv file located in each experiment:

\$folder1/name\_of\_experiment/experiment\_\#/Run\_\#/output/expe-out/out\_jobs.csv \\

Typically I will tar.gz \$folder1 and sftp that back to my computer

\section{In Real Life}
I have things setup on the ac-cluster like so:
\begin{enumerate}
  \item Edit an existing config file, and make sure to rename it before saving it
  \item Set file1=that new config file
  \item Set folder1=a new folder name for this config's set of simulations
  \item Set base=\textasciigrave pwd\textasciigrave \hspace{1mm}  as long as I'm in the basefiles folder
  \item sbatch -p IvyBridge -N1 -n1 -c1 --output=./myBatch.log --export=folder1=\$folder1,file1=\$file1,basefiles=\$base ./myBatch
\end{enumerate}

That's it.  Here is my myBatch script
\begin{code}
#!/bin/bash
source /ac-project/cwalker/start.sh
date
python3 generate_config.py -i /ac-project/cwalker/basefiles/$file1 -o /ac-project/cwalker/experiments/$folder1 --basefiles ${basefiles}  --output-config
date
python3 run-experiments.py -i /ac-project/cwalker/experiments/$folder1  --time 1200 --sim-time-minutes 1200 --socket-start 100003
date

\end{code}

run-experiments will sbatch experiment.sh \\
Here is experiment.sh
\begin{code}
#!/bin/bash
date
echo "in experiment.sh"
hostname
source /ac-project/cwalker/start2.sh
echo "after source"


echo "real_start.py" && date
python3 /ac-project/cwalker/basefiles/real_start.py --path $jobPath --socketCount $socketCount --sim-time $mySimTime
#rm $jobPath/output/expe-out/out*    # do all these rm's to cut down on folder size
#rm $jobPath/output/expe-out/post_out_jobs.csv
#rm $jobPath/output/expe-out/raw_post_out_jobs.csv
#rm -rf $jobPath/output/expe-out/log/
date

\end{code}
real\_start.py is there in basefiles to look at, but here is my start2.sh
\begin{code}
export MODULEPATH=/opt/ohpc/admin/modulefiles:/opt/ohpc/pub/modulefiles:/ac-project/software/modules/linux-centos7-x86_64/Core/linux-centos7-ivybridge && \
export PATH=$PATH:/ac-project/cwalker/Install/bin:/ac-project/cwalker/Downloads/goroot/bin/:/ac-project/cwalker/go/bin/
export LD_LIBRARY_PATH=$LD_LIBRARY_PATH:/ac-project/cwalker/Install/lib:/ac-project/cwalker/Install/lib64
export LMOD_SH_DBG_ON=1
source /ac-project/cwalker/new_env/bin/activate
\end{code}


\end{document}
